\documentclass{beamer}

\newcommand{\course}{CS 1331 Introduction to Object Oriented Programming}
\newcommand{\lesson}{Swing, Part 1 of 5}
\newcommand{\code}{http://www.cc.gatech.edu/~simpkins/teaching/gatech/cs1331/code}

\author[Chris Simpkins]
{Christopher Simpkins \\\texttt{chris.simpkins@gatech.edu}}
\institute[Georgia Tech] % (optional, but mostly needed)

\date[CS 1331]{}


\newcommand{\course}{Introduction to Object-Oriented Programming}
\subject{\course}
\title[CS 1331]{\course}
\subtitle{\lesson}

\author[Chris Simpkins]
{Christopher Simpkins \\\texttt{chris.simpkins@gatech.edu}}
\institute[Georgia Tech]

\date[\lesson]{}

\newcommand{\link}[2]{\href{#1}{\textcolor{blue}{\underline{#2}}}}
\newcommand{\code}{http://www.cc.gatech.edu/~simpkins/teaching/gatech/cs1331/code}

\usepackage{colortbl}

% If you have a file called "university-logo-filename.xxx", where xxx
% is a graphic format that can be processed by latex or pdflatex,
% resp., then you can add a logo as follows:

% \pgfdeclareimage[width=0.6in]{coc-logo}{cc_2012_logo}
% \logo{\pgfuseimage{coc-logo}}

\mode<presentation>
{
  \usetheme{Berlin}
  \useoutertheme{infolines}

  % or ...

 \setbeamercovered{transparent}
  % or whatever (possibly just delete it)
}

\usepackage{tikz}
% Optional PGF libraries
\usepackage{pgflibraryarrows}
\usepackage{pgflibrarysnakes}
\usepackage{pgfplots}
\usepackage{hyperref}
\usepackage{fancybox}
\usepackage{listings}
\usepackage[abbr]{harvard}

\usepackage[english]{babel}
% or whatever

\usepackage[latin1]{inputenc}
% or whatever

\usepackage{times}
\usepackage[T1]{fontenc}
% Or whatever. Note that the encoding and the font should match. If T1
% does not look nice, try deleting the line with the fontenc.


\usepackage{listings}

% "define" Scala
\lstdefinelanguage{scala}{
  morekeywords={abstract,case,catch,class,def,%
    do,else,extends,false,final,finally,%
    for,if,implicit,import,match,mixin,%
    new,null,object,override,package,%
    private,protected,requires,return,sealed,%
    super,this,throw,trait,true,try,%
    type,val,var,while,with,yield},
  otherkeywords={=>,<-,<\%,<:,>:,\#,@},
  sensitive=true,
  morecomment=[l]{//},
  morecomment=[n]{/*}{*/},
  morestring=[b]",
  morestring=[b]',
  morestring=[b]""",
}

\usepackage{color}
\definecolor{dkgreen}{rgb}{0,0.6,0}
\definecolor{gray}{rgb}{0.5,0.5,0.5}
\definecolor{mauve}{rgb}{0.58,0,0.82}

% Default settings for code listings
\lstset{frame=tb,
  language=scala,
  aboveskip=2mm,
  belowskip=2mm,
  showstringspaces=false,
  columns=flexible,
  basicstyle={\scriptsize\ttfamily},
  numbers=none,
  numberstyle=\tiny\color{gray},
  keywordstyle=\color{blue},
  commentstyle=\color{dkgreen},
  stringstyle=\color{mauve},
  frame=single,
  breaklines=true,
  breakatwhitespace=true,
  keepspaces=true
  %tabsize=3
}


% If you wish to uncover everything in a step-wise fashion, uncomment
% the following command:

% \beamerdefaultoverlayspecification{<+->}


\begin{document}

\begin{frame}
  \titlepage
\end{frame}

%------------------------------------------------------------------------
\begin{frame}[fragile]{Outline}


\begin{itemize}
\item Hello, Swing!
\item Event-driven programming
\item Hello, buttons!
\item The observer pattern in Swing
\end{itemize}


\end{frame}
%------------------------------------------------------------------------

%------------------------------------------------------------------------
\begin{frame}[fragile]{Hello, Swing}


Here's a minimal Swing program:
\begin{lstlisting}[language=Java]
import javax.swing.JFrame;

public class HelloSwing {

    public static void main(String[] args) {
        JFrame mainFrame = new JFrame("Hello, Swing!");
        mainFrame.setSize(640, 480);
        mainFrame.setDefaultCloseOperation(JFrame.DISPOSE_ON_CLOSE);
        mainFrame.setVisible(true);
    }
}
\end{lstlisting}

See \link{\code/swing/HelloSwing.java}{HelloSwing.java} and the API documentation for \link{http://docs.oracle.com/javase/7/docs/api/javax/swing/JFrame.html}{JFrame}

\end{frame}
%------------------------------------------------------------------------

%------------------------------------------------------------------------
\begin{frame}[fragile]{{\tt javax.swing.JFrame}}

\vspace{-.1in}
Almost all Swing programs use a {\tt JFrame} for the main GUI window. Here's a high-level recipe for using {\tt JFrame}:
\vspace{-.05in}
\begin{itemize}
\item Instantiate a {\tt JFrame}, passing a title to the constructor
\begin{lstlisting}[language=Java]
        JFrame mainFrame = new JFrame("Hello, Swing!");
\end{lstlisting}
\vspace{-.05in}
\item Set {\tt JFrame}'s initial size (in the next lecture we'll see how to do this in a general way)
\begin{lstlisting}[language=Java]
        mainFrame.setSize(640, 480);
\end{lstlisting}
\vspace{-.05in}
\item Specify what to do when user clicks "x" button on title bar
\begin{lstlisting}[language=Java]
      mainFrame.setDefaultCloseOperation(JFrame.DISPOSE_ON_CLOSE);
\end{lstlisting}
\vspace{-.05in}
\item Add components to the {\tt JFrame} and wire them to listeners
\item Display the {\tt JFrame}
\begin{lstlisting}[language=Java]
        mainFrame.setVisible(true);
\end{lstlisting}
\end{itemize}
\vspace{-.05in}
Today we're using and customizing a {\tt JFrame} object from within another class.  Next lecture we'll create custom subclasses of {\tt JFrame}.

\end{frame}
%------------------------------------------------------------------------


%------------------------------------------------------------------------
\begin{frame}[fragile]{Event-Driven Programming}


So far we've done structured sequential programming where the order of execution is controlled by the programmer.  GUIs use event-driven programming:
\begin{itemize}
\item User is presented with options.
\item User actions (and other actions) fire events.
\item Event handlers execute in response to events.
\item Order of execution is controlled by the order of events, which the programmer does not know in advance.
\end{itemize}


\end{frame}
%------------------------------------------------------------------------

%------------------------------------------------------------------------
\begin{frame}[fragile]{Hello, buttons!}

\vspace{-.1in}
\begin{lstlisting}[language=Java]
public class HelloButtons {
    public static void main(String[] args) {
        JFrame mainFrame = new JFrame("Hello, buttons!");
        mainFrame.setDefaultCloseOperation(JFrame.DISPOSE_ON_CLOSE);
        mainFrame.setLayout(new FlowLayout());

        JButton exitButton = new JButton("Exit");
        exitButton.addActionListener(new ExitListener());

        JLabel counterLabel = new JLabel("Count: 0");
        JButton counterButton = new JButton("Increment count");
        counterButton.addActionListener(
            new CountListener(counterLabel));

        mainFrame.add(exitButton);
        mainFrame.add(counterButton);
        mainFrame.add(counterLabel);
        mainFrame.setSize(300, 275);
        mainFrame.setVisible(true);
    }
}
\end{lstlisting}
\vspace{-.05in}
See \link{\code/swing/HelloButtons.java}{HelloButtons.java}, \link{\code/swing/ExitListener.java}{ExitListener.java} and \link{\code/swing/CountListener.java}{CountListener.java}

\end{frame}
%------------------------------------------------------------------------

%------------------------------------------------------------------------
\begin{frame}[fragile]{The Observer Pattern in Swing}


Three particpants in the observer pattern:
\begin{itemize}
\item An event publisher that fires events
\item An event object that represent the event
\item Event handlers that subscribe to event publishers and receive event objects
\end{itemize}

Practically speaking, firing an event means calling a method on event listeners.  Let's look at a concrete example.




\end{frame}
%------------------------------------------------------------------------

%------------------------------------------------------------------------
\begin{frame}[fragile]{An Event Publisher: {\tt javax.swing.JButton}}


In {\tt HelloButtons.java} we set up an exit button like this:
\begin{lstlisting}[language=Java]
JButton exitButton = new JButton("Exit");
exitButton.addActionListener(new ExitListener());
\end{lstlisting}

\begin{itemize}
\item {\tt JButton}'s {\tt addActionListener} method takes an object that implements the {\tt java.awt.event.ActionListener} interface
\end{itemize}


\end{frame}
%------------------------------------------------------------------------

%------------------------------------------------------------------------
\begin{frame}[fragile]{{\tt java.awt.event.ActionListener}}


\begin{lstlisting}[language=Java]
public interface ActionListener extends EventListener {

    /**
     * Invoked when an action occurs.
     */
    public void actionPerformed(ActionEvent e);

}
\end{lstlisting}

\begin{itemize}
\item Event listeners implement the {\tt ActionListener} interface, which includes only one method.
\end{itemize}


\end{frame}
%------------------------------------------------------------------------

%------------------------------------------------------------------------
\begin{frame}[fragile]{{\tt ExitListener}}


Here's the complete definition for {\tt ExitListener}:
\begin{lstlisting}[language=Java]
import java.awt.event.ActionListener;
import java.awt.event.ActionEvent;

public class ExitListener implements ActionListener {

    public void actionPerformed(ActionEvent e) {
        System.exit(0);
    }
}
\end{lstlisting}

\begin{itemize}
\item Any time the exit button is pressed, {\tt ExitListener}'s {\tt actionPerformed} method is called
\end{itemize}


\end{frame}
%------------------------------------------------------------------------

%------------------------------------------------------------------------
\begin{frame}[fragile]{{\tt CountListener}}

\vspace{-.1in}
Here's the complete definition for \link{\code/swing/CountListener.java}{CountListener.java} (minus imports):
\begin{lstlisting}[language=Java]
public class CountListener implements ActionListener {
    private JLabel countLabel;
    private int count;
    public CountListener(JLabel countLabel) {
        this.countLabel = countLabel;
        count = 0;
    }
    public void actionPerformed(ActionEvent e) {
        count++;
        countLabel.setText("Count: " + count);
    }
}\end{lstlisting}
\vspace{-.1in}
\begin{itemize}
\item {\tt CountListener} keeps a reference to a {\tt JLabel} and an {\tt int} to hold the count
\item When its {\tt actionPerformed} method is called, it updates the count and (re-) sets the text on its {\tt JLabel} with the new count
\end{itemize}
Three objects cooperating: a {\tt JButton}, a {\tt JLabel}, and a {\tt CountListener} (which is-a {\tt ActionListener}) to tie them together.

\end{frame}
%------------------------------------------------------------------------


%------------------------------------------------------------------------
\begin{frame}[fragile]{Closing Thoughts}


\begin{itemize}
\item Event-driven GUI programming requires a shift in thinking.  Putting the user in control means you have to work harder to
\begin{itemize}
  \item handle order dependencies, e.g., by disabling buttons until certain actions are taken, and
  \item guide the user, e.g., by following UI guidelines to maximize familiarity.
\end{itemize}
\item JavaFX is the future, but learning Swing first makes sense because
\begin{itemize}
  \item Swing concepts are present in JavaFX,
  \item JavaFX is much harder to set up and debug (so Swing is better for beginners), and
  \item our goal here is to teach the basics of event-driven GUI programming.
\end{itemize}

\end{itemize}


\end{frame}
%------------------------------------------------------------------------

% %------------------------------------------------------------------------
% \begin{frame}[fragile]{}


% \begin{lstlisting}[language=Java]

% \end{lstlisting}

% \begin{itemize}
% \item
% \end{itemize}


% \end{frame}
% %------------------------------------------------------------------------


\end{document}
