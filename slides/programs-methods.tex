\documentclass{beamer}

\newcommand{\lesson}{Programs and Methods}


\newcommand{\course}{Introduction to Object-Oriented Programming}
\subject{\course}
\title[CS 1331]{\course}
\subtitle{\lesson}

\author[Chris Simpkins]
{Christopher Simpkins \\\texttt{chris.simpkins@gatech.edu}}
\institute[Georgia Tech]

\date[\lesson]{}

\newcommand{\link}[2]{\href{#1}{\textcolor{blue}{\underline{#2}}}}
\newcommand{\code}{http://www.cc.gatech.edu/~simpkins/teaching/gatech/cs1331/code}

\usepackage{colortbl}

% If you have a file called "university-logo-filename.xxx", where xxx
% is a graphic format that can be processed by latex or pdflatex,
% resp., then you can add a logo as follows:

% \pgfdeclareimage[width=0.6in]{coc-logo}{cc_2012_logo}
% \logo{\pgfuseimage{coc-logo}}

\mode<presentation>
{
  \usetheme{Berlin}
  \useoutertheme{infolines}

  % or ...

 \setbeamercovered{transparent}
  % or whatever (possibly just delete it)
}

\usepackage{tikz}
% Optional PGF libraries
\usepackage{pgflibraryarrows}
\usepackage{pgflibrarysnakes}
\usepackage{pgfplots}
\usepackage{hyperref}
\usepackage{fancybox}
\usepackage{listings}
\usepackage[abbr]{harvard}

\usepackage[english]{babel}
% or whatever

\usepackage[latin1]{inputenc}
% or whatever

\usepackage{times}
\usepackage[T1]{fontenc}
% Or whatever. Note that the encoding and the font should match. If T1
% does not look nice, try deleting the line with the fontenc.


\usepackage{listings}

% "define" Scala
\lstdefinelanguage{scala}{
  morekeywords={abstract,case,catch,class,def,%
    do,else,extends,false,final,finally,%
    for,if,implicit,import,match,mixin,%
    new,null,object,override,package,%
    private,protected,requires,return,sealed,%
    super,this,throw,trait,true,try,%
    type,val,var,while,with,yield},
  otherkeywords={=>,<-,<\%,<:,>:,\#,@},
  sensitive=true,
  morecomment=[l]{//},
  morecomment=[n]{/*}{*/},
  morestring=[b]",
  morestring=[b]',
  morestring=[b]""",
}

\usepackage{color}
\definecolor{dkgreen}{rgb}{0,0.6,0}
\definecolor{gray}{rgb}{0.5,0.5,0.5}
\definecolor{mauve}{rgb}{0.58,0,0.82}

% Default settings for code listings
\lstset{frame=tb,
  language=scala,
  aboveskip=2mm,
  belowskip=2mm,
  showstringspaces=false,
  columns=flexible,
  basicstyle={\scriptsize\ttfamily},
  numbers=none,
  numberstyle=\tiny\color{gray},
  keywordstyle=\color{blue},
  commentstyle=\color{dkgreen},
  stringstyle=\color{mauve},
  frame=single,
  breaklines=true,
  breakatwhitespace=true,
  keepspaces=true
  %tabsize=3
}


% \beamerdefaultoverlayspecification{<+->}

\begin{document}

\begin{frame}
  \titlepage
\end{frame}

%------------------------------------------------------------------------
\begin{frame}[fragile]{The Anatomy of a Java Program}


It is customary for a progarmmer's first program in a new language to be ``Hello, World.''  Here's our \link{\code/HelloWorld.java}{{\tt HelloWorld.java}} program:
\begin{lstlisting}[language=Java]
public class HelloWorld {
    public static void main(String[] args) {
        System.out.println("Hello, world!");
    }
}
\end{lstlisting}
\vspace{-.1in}
\begin{itemize}
\item The first line declares our {\tt HelloWorld} class.  {\tt class} is the syntax for declaring a class, and prepending with the {\tt public} modifer means the class will be visible outside {\tt HelloWorld}'s package.
\item Because we didn't declare a package explicitly, {\tt HelloWorld} is in the {\it default} package.  More on packages in a future lectrue.
\item The code between the curly braces, {\tt \{ ... \}} define the contents of the {\tt HelloWorld} class, in this case a single method, {\tt main}
\end{itemize}

\end{frame}
%------------------------------------------------------------------------

%------------------------------------------------------------------------
\begin{frame}[fragile]{{\tt public static void main(String[] args)}}


In order to make a class executable with the {\tt java} command, it must have a main method:
\begin{lstlisting}[language=Java]
public static void main(String[] args) { ... }
\end{lstlisting}
\vspace{-.1in}
\begin{itemize}
\item The {\tt public} modifier means we can call this method from outside the class.
\item The {\tt static} modifer means the method can be called without instantiating an object of the class.  Static methods (and variables) are sometimes called {\it class} methods.
\item {\tt void} is the return type.  In particular, main returns nothing.  Sometimes such subprograms are called {\it procedures} and distinguished from {\it functions}, which return values.
\item After the method name, {\tt main}, comes the parameter list.  {\tt main} takes a single parameter of type {\tt String[]} - an array of {\tt String}s.  {\tt args} is the name of the parameter, which we can refer to within the body of {\tt main}
\end{itemize}

\end{frame}
%------------------------------------------------------------------------

%------------------------------------------------------------------------
\begin{frame}[fragile]{Methods}

These {\tt main} method is a special method that is used as the entry point for a Java program.  We can define other methods as well.  Consider this method from \link{\code/basics/NameParser.java}{NameParser.java}:

\begin{lstlisting}[language=Java]
public static String extractLastName(String name) {
  int commaPos = name.indexOf(",");
  int len = name.length();
  String lastName = name.substring(0, commaPos).trim();
  return lastName;
}
\end{lstlisting}
Similar to our {\tt main} method but:
\begin{itemize}
\item {\tt return}s a {\tt String}
\item takes a single parameter of type {\tt String}
\end{itemize}

\end{frame}
%------------------------------------------------------------------------

%------------------------------------------------------------------------
\begin{frame}[fragile]{Method Parameters}

In this method:
\begin{lstlisting}[language=Java]
public static String extractLastName(String name) {
  int commaPos = name.indexOf(",");
  int len = name.length();
  String lastName = name.substring(0, commaPos).trim();
  return lastName;
}
\end{lstlisting}
{\tt name} is a parameter, a local scope variable within the {\tt extractLastName} method.  It is bound to a value when the method is called.  In:

\begin{lstlisting}[language=Java]
String lastName = extractLastName(fullName);
\end{lstlisting}
we say that {\tt fullName} is the {\it argument} to this invocation of the {\tt extractLastName} method.

\end{frame}
%------------------------------------------------------------------------


%------------------------------------------------------------------------
\begin{frame}[fragile]{Methods as Expressions}

Methods that return values are expressions which can be used anywhere a type-equivalent expression can be used.  Given:

\begin{lstlisting}[language=Java]
public static int add(int a, int b)
\end{lstlisting}
This:
\begin{lstlisting}[language=Java]
x + (x + y)
\end{lstlisting}
is equivalent to this:
\begin{lstlisting}[language=Java]
x + add(x, y)
\end{lstlisting}
See \link{\code/basics/Expressions.java}{Expressions.java}.

\end{frame}
%------------------------------------------------------------------------


% %------------------------------------------------------------------------
% \begin{frame}[fragile]{}

% \begin{itemize}
% \item
% \end{itemize}

% \begin{lstlisting}[language=Java]

% \end{lstlisting}


% \end{frame}
% %------------------------------------------------------------------------

\end{document}
